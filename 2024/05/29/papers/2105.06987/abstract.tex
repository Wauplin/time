Ensembles of machine learning models yield improved system performance as well as robust and interpretable uncertainty estimates; however, their inference costs may often be prohibitively high.
\emph{Ensemble Distribution Distillation} is an approach that allows a single model to efficiently capture both the predictive performance and uncertainty estimates of an ensemble. For classification, this is achieved by training a Dirichlet distribution over the ensemble members' output distributions via the maximum likelihood criterion. Although theoretically principled, this criterion exhibits poor convergence when applied to large-scale tasks where the number of classes is very high.
In our work, we analyze this effect and show that for the Dirichlet log-likelihood criterion classes with low probability induce larger gradients than high-probability classes. This forces the model to focus on the distribution of the ensemble tail-class probabilities.
We propose a new training objective which minimizes the reverse KL-divergence to a \emph{Proxy-Dirichlet} target derived from the ensemble. This loss resolves the gradient issues of Ensemble Distribution Distillation, as we demonstrate both theoretically and empirically on the ImageNet and WMT17 En-De datasets containing 1000 and 40,000 classes, respectively.